\setlength{\fboxrule}{3pt} 
\setlength{\fboxsep}{5pt} 
\FloatBarrier
\subsection{Einstein Telescope Science Goals}

Some three hundred years after Galileo observed the Jovian 
satellites, the twentieth century heralded a new era in 
observational astronomy with  radio and microwave antennas, 
gamma- and X-ray detectors, which revolutionized astronomy and 
opened the entire electromagnetic window for observing 
and understanding the Universe. A remarkable revelation 
coming from these observations is that about 96 percent of 
our Universe is invisible and that gravitational interaction 
powers the most luminous and spectacular objects and phenomena 
such as quasars, gamma-ray bursts, ultra luminous X-ray sources, 
pulsars, and the evolution of the early Universe.

Einstein's theory of gravity predicted that dynamical systems 
in strong gravitational fields will release vast amounts of 
energy in the form of gravitational radiation. This radiation 
has the potential to open a new window on the Universe, 
complementing the electromagnetic window. Russell Hulse 
and Joe Taylor were awarded the 1993 Nobel Prize in Physics for 
the discovery of a binary consisting of two neutron stars in 
close orbit in which indirect evidence for the emission of
gravitational waves was found. 

Interferometric gravitational wave (GW) detectors that are currently 
taking data and advanced detectors that will be built over 
the next ten years will be the first steps in establishing 
the field of gravitational astronomy through their detection 
of the most luminous sources such as the merger 
of binary neutron stars and black holes.  Einstein Telescope will 
make it possible to observe a greater variety of phenomena, and 
provide a new tool for expanding our knowledge of fundamental 
physics, cosmology and relativistic astrophysics. Is the 
nature of gravitational radiation as predicted by Einstein's 
theory? Are black hole spacetimes uniquely given by the Kerr geometry?
Do event horizons always form around gravitationally collapsing matter?
%Are naked singularities the final state of gravitational collapse? 
How did the black holes at galactic nuclei form? What were the 
physical conditions in the very early Universe? What is the 
nature of quantum gravity and what is the origin of space and 
time? Are there really ten spatial dimensions? These are 
some key questions at the forefront of physics that future GW 
observations might shed some light on.
\FloatBarrier
\subsubsection{Fundamental physics}
Astronomical sources of gravitational waves are essentially 
systems where gravity is extremely strong and often 
characterized by relativistic bulk motion of massive objects. 
The emitted radiation carries an uncorrupted signature of the 
nature of the space-time geometry and is therefore an invaluable 
tool to understand the behaviour of matter and geometry in 
extreme conditions of density, temperature, magnetic fields 
and relativistic motion. Here are some examples of how GW 
observations can impact fundamental physics. 

In Einstein's theory, gravitational radiation travels at the 
speed of light and has two polarization states. In alternative 
theories of gravity one or both of these properties might not
hold, owing to the presence of massive gravitons, or a scalar 
field mediating gravity in addition to the tensor field. Experimental 
tests of gravity, as well those afforded by the data from the 
Hulse-Taylor binary, are consistent with both Einstein's theory 
and one of its alternatives called the Brans-Dicke theory. 
Gravitational wave detectors will bring these theories 
vis-a-vis observations that could decisively rule out one 
or the other. 

According to Einstein's gravity the space-time in the vicinity 
of black holes is described by a unique geometry called the 
Kerr solution. Observation of the radiation from the in-fall 
of stellar-mass black holes into intermediate-mass black holes 
will make it possible to test such uniqueness theorems. X-ray 
astronomy has provided firm indirect evidence that intense 
sources of x-rays may well host a black hole. An unambiguous 
signature of the black hole geometry, however, could eventually 
be provided by the detection of black hole quasi-normal modes: 
gravitational radiation with characteristic frequencies and decay 
times. 
Failure to detect such radiation from, for example, a newly 
formed black hole would mean that gravity is more exotic than 
what we currently believe, 
%(e.g., gravitational collapse might lead to entities called naked singularities) 
and might reveal new phases of matter at extremely high densities. 

The most attractive versions of string theory require a ten- or
eleven-dimensional space-time, far larger than what we perceive. 
In certain phenomenological models at the interface of string 
theory and cosmology, what we perceive as a four-dimensional 
Universe could be one part, or ``brane'', within a higher 
dimensional ``bulk'' Universe. The extra spatial dimensions may 
be compact and sub-millimetre-scale, 
%(so-called ``Large Extra Dimensions'') 
or even macroscopically large, if their geometry has 
properties known as ``warping''. The key feature of brane-world 
theories is that gravitational interactions, and in particular 
gravitational waves, propagate in the bulk, while other interactions 
are restricted to the brane, which partly explains why gravity is 
so weak. 
%Brane world models predict a specific signature in the spectrum of gravitational waves. Future ground- and space-based gravitational wave detectors offer the exciting possibility of observing radiation from the bulk and thereby explore whether the Universe has large extra dimensions.
% SOME STUFF ABOUT HOW TO TEST STRING THEORY \& BRANES

% OR COMMENTS ON NEW PHYSICS IN GENERAL IN RELATION TO
% STOCHASTIC?
\FloatBarrier
\subsubsection{Relativistic Astrophysics}
Astronomy has revealed a Universe full of diverse and exotic
phenomena which remain enigmas decades after their discovery. 
Supernovae are the end-states of stellar evolution, resulting in 
gravitational collapse followed by a huge explosion of in-falling matter.
Gamma-ray bursts are intense sources of gamma radiation that last 
only a few seconds to minutes yet emit more energy than a star does 
in its entire lifetime. 
Radio pulsars are compact objects as massive as the Sun but only 
\nomenclature[gC]{Compact star}{ Throughout this document a compact star
stands for a neutron star or a black hole}
about 10 km in size, and the regularity of their radio pulses rivals 
the best atomic clocks in the world. 
Transient radio sources thousands of light years away 
are associated with magnetic fields so strong that the emitted radiation 
could breakdown terrestrial radio stations. 
%
For each of these objects the source is believed to be couched in 
dense environs and strong gravitational fields and, therefore, 
is a potential source of gravitational radiation. For example, gamma-ray 
bursts could be produced by colliding neutron stars which are 
electromagnetically invisible for most of their lives but are very 
powerful emitters of GW. Transient radio sources could be the 
result of quakes in neutron stars with concomitant emission of GW. 
Observing such `multi-messengers' (sources that are strong emitters of 
both EM and GW radiation) will help understand phenomena that 
have remained puzzles for decades.

The centre of every galaxy is believed to host a compact 
object a million to a billion times as massive as our Sun, 
a powerful emitter of optical, radio and other radiation. 
What is the nature of this object? How and when it form? 
Did it form from small 100 solar mass seeds and then grow 
by accreting gas and other compact objects? What is its 
relation to the size of the galaxy as a whole? These are some 
of the questions which a model of the formation of structure in 
the Universe must answer.  While electromagnetic 
observations have provided valuable data, GW observations can 
help address some of the key questions on the formation and 
nature of these objects. 

Future gravitational wave detectors will also be sensitive to 
a population of sources at very high red-shifts, helping us
study cosmological evolution of sources, the history of star 
formation and its dependence on the matter content of 
the Universe, and development of large-scale structure in the 
Universe.

\subsubsection{Cosmology}
The twentieth century was the golden age of cosmology. With 
the advent of radio and microwave astronomy it was possible to 
finally address key questions about the origin of the 
Universe. 
%and if it really started in a big bang. 
The cosmic microwave background (CMB) is a relic radiation 
from the hot ``Big Bang'' that is believed to have been the 
initial condition for primordial nucleosynthesis. Since the early 
Universe was very dense, this radiation was in thermal 
equilibrium with matter for about 380,000 years after the Big 
Bang and cannot directly reveal the conditions in the very 
early phases of the Universe's history. The most direct way of 
observing the primaeval Universe is via the gravitational 
window with a network of two or more detectors. From fairly 
general assumptions one can predict the production of 
a stochastic background of GW in the early Universe, which travel 
to us unscathed as a consequence of their weak coupling to matter. 

The most amazing aspect of the Universe is that only about 
4\% of its energy density appears in the form of visible matter, 
the rest being dark matter and dark energy. 
In order to understand the behaviour of these `dark' contents 
it is necessary to have a standard candle -- a population of 
sources whose distance from Earth can be inferred from their 
luminosity. 
Compact binaries are an astronomer's ideal standard candle: 
By measuring the signature of the gravitational radiation they emit, 
it is possible to infer their intrinsic parameters (e.g.\ the 
masses and spins of the component objects) and accurately 
deduce their luminosity distance. In fact, compact binaries
eliminate the need to build a cosmic distance ladder -- the
process by which standard candles at different distances are 
calibrated in astronomy since there is no source that is 
suitable at all distances.

The synergy of multi-messenger astronomy is nowhere more
apparent than in the use of standard sirens of gravity to
test the concordance model of cosmology.  ET might detect 
several hundred compact binary coalescence events each year 
\nomenclature[gC]{Compact binary}{ A compact binary is an astronomical
binary consisting of a pari of compact stars, i.e.\ a neutron star or
a black hole}
in coincidence with short-hard gamma-ray bursts, provided, of 
course, the two are related.  While gravitational observations would
provide an unambiguous measure of the luminosity distance,
the host galaxy of the GRB could be used to measure the 
redshift. By fitting the observed population to a cosmological 
model, it will be possible to measure the Hubble parameter, 
dark matter and dark energy densities, as well as the 
dark energy equation-of-state parameter.

The early history of the Universe may have witnessed 
several phase transitions as the temperature decreased 
through the energy scales of a Grand Unified Theory (GUT) 
and electroweak symmetry-breaking, and eventually to the 
current state in which we see four different fundamental 
interactions. During some phase transitions, cosmic strings 
form as one-dimensional topological defects at the 
boundaries of different phases. Vibrations of these strings 
at the speed of light can sometimes form a kink which emits a 
burst of gravitational radiation. The spectrum of such radiation 
has a unique signature which can help us detect cosmic strings
and measure their properties, and thus provide a glimpse of the 
Universe as it underwent phase transitions.

Perhaps the most exciting discovery of the new window will 
be none of the above. If the astronomical legacy is any example, 
gravitational astronomy should unveil phenomena and 
sources never imagined in the wildest theories -- a possibility
of any new observational tool. 

\FloatBarrier
\subsection{Sources of gravitational waves in ET}

The goal of this Section is to give an overview of the sources
expected to be observed by ET and problems addressed in
the context of the Design Study. A very brief
introduction to gravitational waves is given in Box \ref{box:response}.

\nomenclature[sh]{$h$}{ Gravitational wave amplitude, usually denoting
the detector response}
%\begin{wrapfigure}{r}{0.50\textwidth}
%\begin{boxedminipage}{0.50\textwidth}
\etbox{i}{box:response}{Response of a detector}
{Gravitational waves are described by a second rank tensor 
$h_{\alpha\beta}$, which, in a suitable coordinate system and gauge,
has only two independent components $h_+$ and $h_\times,$ $h_{xx}=-h_{yy}=h_+,$
$h_{xy} = h_{yx} = h_\times$, all other components being zero.
A detector measures only a certain linear combination of the
two components, called the {\em response} $h(t),$ given by
\begin{equation}
h(t) = F_+(\theta,\, \varphi,\, \psi) h_+(t) +
       F_\times(\theta,\, \varphi,\, \psi) h_\times(t),
\label{eq:response}
\end{equation}
where $F_+$ and $F_\times$ are the detector antenna pattern functions,
$\psi$ is the polarization angle, and $(\theta,\,\varphi)$ are angles
describing the location of the source on the sky. The angles are
all assumed to be constant for a transient source but time-dependent 
for sources that last long enough so that the Doppler modulation 
of the signal due to the relative motion of the source and detector
cannot be neglected. 
}
%\end{boxedminipage}
%\end{wrapfigure}
\FloatBarrier
\subsubsection{Compact binary coalescences}

A compact binary, consisting of neutron stars (NS) and/or black holes (BH),
evolves by emitting gravitational radiation which extracts the rotational 
energy and angular momentum from the system, thereby leading to an inspiral 
of the two bodies towards each other.  The dynamics of a compact binary consists
of three phases: (i) The {\it early inspiral phase} in which the
system spends 100's of millions of years and the luminosity
in GW is rather low and the dynamics can be solved using approximation
methods - the most popular being the post-Newtonian (PN) approximation
(see Box \ref{box:PN}). 
\nomenclature[aPN]{PN}{ Post-Newtonian, an approximation to Einstein's field
equations}

The inspiral signal has a characteristic shape,
with slowly increasing amplitude and frequency and is called a {\it chirp} 
waveform.  A binary signal that chirps (i.e.\ its frequency changes 
perceptibly during the course of observation) is an astronomer's {\it standard 
candle} \cite{Schutz86} (see below) and by observing the radiation from
a {\it chirping} binary we can measure the luminosity distance to the source.
(ii) The {\it plunge} phase when the two stars are moving at a 
third of the speed of light and experiencing strong gravitational fields with 
the gravitational potential being $\varphi = GM/Rc^2 \sim 0.1.$ This
phase requires the full non-linear structure of Einstein's equations 
as the problem involves strong relativistic gravity, tidal deformation
(in the case of BH-BH or BH-NS) and disruption (in the case of BH-NS and NS-NS)
and has only recently been solved by numerical relativists (see below). 
Analytical solutions based on resummation of the PN series have been very
successful in describing the merger phase.
(iii) The {\it merger,} or {\it ringdown,} phase when the 
two systems have merged to form either a NS or BH, settling down to a 
quiescent state by radiating the deformations inherited during the merger. 
The emitted radiation can be computed using perturbation theory and 
gives the quasi-normal modes (QNM) of BH and NS. The QNM carry 
a unique signature that depends only on the mass and spin angular momentum
in the case of BH, but depends also on the equation-of-state (EOS) 
of the material in the case of NS.  

The adiabatic inspiral, during which the signal is approximated by (\ref{eq:amps}),
is followed by the merger of the compact objects, leading to the formation of a single
black hole. This black hole then undergoes a rapid `ringdown' as it settles down 
to a quiescent state. Following breakthroughs in 2005
\cite{Pretorius05,Campanelli:2005dd,Baker:2005vv}, 
it is now possible to numerically solve the
full Einstein equations for the last orbits, merger and ringdown of
\nomenclature[aLSO]{LSO}{ Last stable orbit}
comparable mass black-hole-binary systems, and to calculate the
emitted GW signal. Subsequent dramatic progress has lead both to
simulations of rapidly increasing numerical accuracy and physical
fidelity, and to the inclusion of larger numbers of GW cycles before 
merger, allowing full GR waveforms to be in principle useful for searches of
black-hole binaries of ever lower mass; see Fig. 3 in \cite{Hannam:2009rd}.
The inclusion of merger and ringdown dramatically increases the signal-to-noise
ratio, leading to a much larger distance reach than one would have with the
inspiral signal alone. As we shall see below, having observational access to 
these later stages of the coalescence 
process will lead to key insights into the structure of neutron stars; in the
case of black holes it will open up the genuinely strong-field dynamics 
of spacetime. 

\paragraph{Standard Sirens of Gravity}

Cosmologists have long sought for standard candles that can 
work on large distance scales without being dependent on the 
lower rungs of cosmic distance ladder. In 1986, Schutz \cite{Schutz86} 
pointed out that gravitational astronomy can 
provide such a candle, or, more appropriately, a {\em standard 
siren}, in the form of a chirping signal from the coalescence 
of compact stars in a binary.
The basic reason for this is that the gravitational-wave 
amplitude depends only on the ratio of a certain combination of the
binary masses and the luminosity distance. For chirping signals 
observations can measure both the amplitude of the signal and 
the masses very accurately and hence infer the luminosity distance.
\nomenclature[aQNM]{QNM}{Quasi-normal mode}

%\begin{wrapfigure}{r}{0.50\textwidth}
%\begin{boxedminipage}{0.50\textwidth}
\etbox{i}{box:PN}{Post-Newtonian description of the inspiral signal}
{The adiabatic evolution of a compact binary, during which the 
emission of gravitational waves causes the component stars 
of the system to {\em slowly} spiral-in towards each other, 
can be computed very accurately using the post-Newtonian 
(PN) expansion of the Einstein equations. Currently, the 
dissipative dynamics is known \cite{BLANCHETREF} to order
$O(v^7/c^7),$ where $v$ is the characteristic velocity in 
the system.  
\\[10pt]\indent
\nomenclature[sM]{$M$}{ total mass of a binary or a black hole}
\nomenclature[sn]{$\nu$}{ Symmetric mass ratio; for a binary composed of
compact stars of masses $m_1$ and $m_2$ the symmetric mass ratio is
$\nu=m_1m_2/(m_1+m_2)^2$}
For a binary consisting of two stars of masses $m_1$ and 
$m_2$ (total mass $M\equiv m_1+m_2$ and symmetric mass ratio 
$\nu\equiv m_1m_2/M^2$) and located at a distance $D_{\rm L}$, 
\nomenclature[sD]{$D_{\rm L}$}{ Luminosity distance to the source}
the dominant gravitational wave amplitudes are 
\begin{eqnarray}
h_+(t) & = & \frac{2\nu M}{D_{\rm L}}(1 + \cos^2\iota)
\left [M\omega(t;t_0,M,\nu) \right ]^{\frac{2}{3}} \cos \left [2\Phi(t; t_0, M,\nu) + \Phi_0 \right ],\\
h_\times(t) & = & \frac{2\nu M}{D_{\rm L}}2\cos\iota\,
\left [M\omega(t;t_0,M,\nu) \right ]^{\frac{2}{3}} \sin \left [2\Phi(t;t_0, M,\nu) + \Phi_0 \right ],
\label{eq:amps}
\end{eqnarray}
where $\iota$ is the angle of inclination of the binary's orbital angular
momentum with the line-of-sight, $\omega(t)$ is the angular velocity 
of the equivalent one-body system around the binary's centre-of-mass and 
$\Phi(t;\, t_0,M,\nu)$ is the corresponding orbital phase. Parameters 
$t_0$ and $\Phi_0$ are constants giving the epoch of merger and the 
orbital phase of the binary at that epoch, respectively. 
\\[10pt]\indent
The above expressions for $h_+$ and $h_\times$ are the dominant 
terms in what is essentially a PN perturbative series, an 
approximation technique that is used in solving the Einstein 
equations as applied to a compact binary.  This dominant amplitude
consists of only twice the orbital frequency. Higher order amplitude 
corrections contain other harmonics (i.e.\ phase terms consisting of 
$k\,\Phi(t),$ $k=1,3,4,\ldots$).  Also, the above expressions are 
written down for a system consisting of non-spinning components on a 
quasi-circular orbit. In reality, we can assume neither to be 
true. Waveforms for binaries on an eccentric inspiral orbit are 
known as are those with spin effects but we shall not discuss them here.}
%\end{boxedminipage}
%\end{wrapfigure}


The detector response depends only on a
small number of signal parameters, which can all be measured
either directly or indirectly. The signal is insensitive to
the composition of the component stars and there is
no complicated modelling that involves the structure of the
stars or their environments. Consequently, the measurement
of the luminosity distance is precise, except 
for statistical errors, whose magnitude depends
on the signal-to-noise ratio (SNR), and systematic
errors due to weak gravitational lensing. We will discuss
the relative magnitude of these errors later on.

\etbox{i}{box:effective distance}{Effective Distance}
{Substituting the expressions given in Eq.\,(\ref{eq:amps}) in Box \ref{box:PN}
for $h_+$ and $h_\times$ in Eq.\,(\ref{eq:response}), we get
\begin{equation}
h(t) = \frac{2\nu M}{D_{\rm eff}}\, \left [ M\omega(t)\right]^{\frac{2}{3}} \cos[2\Phi(t) + \Phi_0'].
\label{eq:response1}
\end{equation}
Here $D_{\rm eff}$ is the effective distance to the binary, which
is a combination of the true luminosity distance and 
the antenna pattern functions, and $\Phi_0'$ is a constant phase
involving the various angles,
\begin{equation}
D_{\rm eff} \equiv \frac{D_{\rm L}}{ \left [ F_+^2 (1+\cos^2\iota)^2 +
       4 F_\times^2 \cos^2\iota \right ]^{1/2}}, \quad
\Phi_0' \equiv \Phi_0 + \arctan \left [-\frac{2 F_\times\cos\iota }{F_+ (1+\cos^2\iota)}\right ].
\label{eq:response2}
\end{equation}
Note that $D_{\rm eff} \ge D_{\rm L}.$
In the case of non-spinning binaries on a quasi-circular orbit,
therefore, the signal is characterized by nine parameters in all, 
$(M, \nu, t_0, \Phi_0, \theta, \varphi, \psi, \iota, D_{\rm L}).$
\\[10pt]\indent
Since the phase $\Phi(t)$ of the signal is known 
to a high order in PN theory, one employs matched filtering to 
extract the signal and in the process measures the two mass 
parameters $(M,\, \nu)$ (parameters that completely determine 
the phase evolution) and the two fiducial parameters $(t_0,\, \Phi_0).$
In general, the response of a single interferometer will not be 
sufficient to disentangle the luminosity distance from the angular 
parameters. However, EM identification (i.e.\ electromagnetic, 
especially optical, identification) of the source will determine
the direction to the source, still leaving three unknown parameters
$(\psi,\, \iota,\, D_{\rm L})$. If the signal is a transient, as 
would be the case in ground-based detectors, a network of three 
interferometers will be required to measure all the unknown 
parameters and extract the luminosity distance.}

Although the inspiral signal from a compact binary is a standard 
siren, there is no way of inferring from it the red-shift to a source. 
\nomenclature[sz]{$z$}{ Cosmological redshift to the source}
The mappings $M \rightarrow (1+z) M$, $\omega \rightarrow \omega/(1+z),$ 
and $D_{\rm L} \rightarrow (1+z) D_{\rm L},$ in Eq.\,(\ref{eq:amps}), 
leave the signal invariant.  Note that a source with an intrinsic 
(i.e.\ physical) total mass $M_{\rm phys.}$ at a red-shift $z$ will 
appear to an observer to be a binary of total mass $M_{\rm obs.}
=(1+z)M_{\rm phys.}$. One must optically identify
the host galaxy to measure its red-shift. Thus, there is
synergy in GW and EM observations which can make precision
cosmography possible, without the need to build a cosmic distance 
ladder. Later in this document we will see how to exploit
compact binaries for fundamental physics and cosmography.



%
