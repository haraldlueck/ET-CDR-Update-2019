%
% this file is inserted through \input both in the Optics.tex and in the Conclusion.tex files
%
%\emph{Author(s): \textbf{A.\ Freise and H.\ Lueck} \\}
The main impact of the optical design on the cost is via the required infrastructure in terms of the vacuum system,
tunnels and caverns. In addition the optical system yields direct costs 
for the high-quality optics and the auxiliary optical systems with their mechanical and electrical support systems
The core elements of the interferometers are the test masses which need to be of considerable size and weight 
and must fulfil extremely stringent requirement for bulk and surface qualities.

The maximum available size for Suprasil~3001 today is the size of the 40\,kg test masses for advanced LIGO. The heavier 
substrates envisaged for ET can presently only be made either of Suprasil~3002 which has inhomogeneities in the direction of the 
beam or from fused pieces of thinner Suprasil~3001. The price quoted by Heraeus for a substrate of 600\,mm in diameter 
and 400\,mm in thickness made of composite Suprasil~3001 is \euro700k which corresponds approximately to a 
price of \euro3.5k/kg. The end test masses can be made out of Suprasil 312 with a considerable lower cost per kilogramme,
an estimate based on quotes for Advanced Virgo optics gives \euro1.6k/kg.

At present silicon is only available in Chzochalski grown crystals to a size of up to 450\,mm.%, the maximum resistance available for this type is ???
In Floating zone crystals the maximum size at present is 200\,mm. The interest of the semiconductor industry in ultra-pure crystals is 
very limited and will most likely not drive the development of bigger size single crystals.
Costs of silicon per kg are based on a quote by Simat and are comparable to the one of ultra-pure fused silica.

The cost estimates for polishing and coating of the mirrors are based on similar costs related to Advanced Virgo, up-scaled
to include ion-beam polishing.

Further included in the optics cost are the main laser systems, optical benches in air and in vacuum for injecting the beam from the laser
into the detectors, similar benches and components for extracting and detecting beams from the interferometers as well as
the mechanical, optical and electronic components required for interferometer control purposes. We further count the
cost for special optical subsystems such as the squeezed light source and the thermal compensation systems. These systems
will be directly based on current systems and thus the costs have been estimated from the data for current and advanced 
detectors.

The total cost for the installation of the optical components of the first detector (two interferometers plus spares) has
been computed as \euro37.6M. The total optics cost for all three detectors is estimated to be \euro60M. The nonlinear increase results
from the fact that spare parts can be shared between detectors. The cost of the first detector is dominated by the main
optics which account for \euro20.4M. The high-power laser system at 1064\,nm and one medium power system at 1550\,nm
wavelength cost together \euro7M. The input and output optics require together another \euro7M. We also include 
\euro2.5M for control electronics and \euro600k for the thermal compensation system.
