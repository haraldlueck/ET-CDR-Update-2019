\FloatBarrier
\subsection{Parametric instability}
%\emph{Author: \textbf{Kazuhiro Yamamoto, Daniel Heinert, Sergey E. Strigin}\\}
\label{sec:opt_layout_PI}

Parametric instability is one of the important issues in future
interferometric detectors \cite{Braginsky2001}. Such interferometers
have at least a few km length arm cavities. The spectral distance between 
optical modes in these long cavities are on the order of 10 kHz.
This value is comparable with the frequencies of elastic modes of
the cavity mirrors. In such cases, the parametric instability
becomes a serious problem in the stable operation of
interferometers. A small thermally driven elastic vibration
modulates the light and excites the transverse optical modes of the
cavity which is called Stokes modes. These excited optical modes apply modulated radiation
pressure on the mirrors. This makes the amplitude of the elastic
modes larger. At last, the elastic modes and optical Stokes modes which is 
different from injected beam oscillate largely. This is the parametric instability.

The formula of the parametric instability of a Fabry-Perot cavity
(without anti-Stokes modes) is derived in Ref. \cite{Braginsky2001}.
If the parametric gain $R$ of an elastic mode is larger than unity, that
mode is unstable. The formula of $R$ is
\begin{equation}
R = \sum_{\rm optical\ mode} \frac{4PQ_{\rm m}Q_{\rm
o}}{McL{\omega_{\rm m}}^2} \frac{\Lambda_{\rm o}}{1+\Delta
\omega^2/{\delta_{\rm o}}^2}, \label{R}
\end{equation}
where $P,Q_{\rm m},Q_{\rm o},M,c,L,\omega_{\rm m},\Delta\omega$,
and $\delta_{\rm o}$ are the optical power in the cavity, the
Q-values of the elastic and optical modes, the mass of the mirror,
the speed of light, the cavity length, the angular frequency of
the elastic mode, the angular frequency difference between the
elastic and optical modes, and the half-width angular frequency of
the optical mode, respectively. The value $\Lambda_{\rm o}$
represents the spatial overlap between the optical and elastic
modes. If the shapes of the optical and elastic modes are similar,
$\Lambda_{\rm o}$ is on the order of unity. If the shapes are not
similar, $\Lambda_{\rm o}$ is almost zero. When the shapes and
frequencies of the optical and elastic modes are similar
($\Lambda_{\rm o} \sim 1, \Delta \omega \sim 0$), $R$ will become
several thousands in second generation projects; Advanced LIGO and 
Advanced Virgo. In these projects, the effect of parametric
instability is a serious problem \cite{Ju2006a,Ju2006b}. 

Here, the parametric instability of the Einstein Telescope
(ET) interferometer is discussed. This instability depends on the
specification of the interferometer. However, details of
the design are considered and discussed now. Therefore, outlines of
the instability of the ET interferometer, preliminary calculation results, 
and future work are shown. In
order to simplify the discussion, only the instability of a
Fabry-Perot cavity is considered. The effect of power and signal
recycling (or resonant sideband extraction) techniques are not taken into
account.

At first, we consider the parametric instabilities in Advanced LIGO
as an example of the second generation.
%In order to calculate $\omega_{\rm
%m}$ and $\Lambda_{\rm o}$ for the instability evaluation, we used
%ANSYS, which is a software application for a finite-element
%method.
At second, the parametric instability in ET is
discussed (outline of instability in ET and the preliminary results of calculation 
are shown). At third, how to suppress the instability of the ET
arm cavity is considered. The last part is devoted to a
summary (and future work).

\textbf{Parametric instability of the second generation interferometers}
\nopagebreak

As example of the second generation interferometers, the Advanced LIGO is considered. The specifications of 
Advanced Virgo are similar. Table \ref{tab:specification2} gives the specifications of Advanced
LIGO in Refs. \cite{Ju2006a,Ju2006b,Zhao2005} (after these references, the
specifications of Advanced LIGO were changed). 
  
Study of the instability in Advanced LIGO by a group at the
University of Western Australia \cite{Ju2006a,Ju2006b} is reviewed
briefly. They investigated what happens when the curvature of a
mirror is changed. The curvature of the other mirror is the
default value given in Table \ref{tab:specification2}. Reference
\cite{Ju2006b} shows the curvature dependence of the unstable mode
number. 
The number of the unstable modes of mirror cavity is between 20 and 60 
(it must be noted that this number increases if the higher elastic modes are taken into account). 
Reference \cite{Ju2006a} shows that the
maximum of $R$ in the various elastic modes strongly depends on
the mirror curvature. Even a shift of only a few meters in the
mirror curvature causes a drastic change of the maximum $R$. The
requirement of the accuracy in the mirror curvature in Advanced
LIGO is difficult to be achieved.

\textbf{Parametric instability of Einstein Telescope: Specifications of Einstein Telescope}
\nopagebreak

Here, the parametric instability of the ET is
considered. The details of the ET interferometer are not decided
now. Here, we adopt parameters of ET-C \cite{Hild2010a}. 
These parameters are summarized in Table
\ref{tab:specification2}. Parameters of ET-D \cite{Hild2010b} (this is newer than ET-C) 
are similar to those of ET-C. 
The difference does not change the paremetric instability. The comparison with old sensitivity,
ET-B \cite{HildETconventional,Yamamoto2009}, is also discussed here.  
\begin{table}[h]
\begin{center}
\begin{tabular}{llll}
\hline
\hline
 &Advanced LIGO&ET-C-HF&ET-C-LF\\
\hline
Laser beam profile&Gaussian&LG33&Gaussian\\
Wavelength($\lambda$)&1064 nm&1064 nm&1550 nm\\
Cavity length($L$)&4000 m&10000 m&10000 m\\
Front mirror curvature radius($R_1$)&2076 m&5643 m&6109 m\\
End mirror curvature radius($R_2$)&2076 m&5643 m&6109 m\\
Beam radius at the mirrors($w_{\rm i}$)&60 mm&72.5 mm&120 mm\\
Finesse (${\cal F}$) & 1250 & 850 & 850 \\
Power in a cavity($P$) &0.83 MW&3 MW&18 kW\\
Mirror material&Fused silica&Fused silica&Silicon\\
Sound velocity in mirror ($v$) &5.7 km/s&5.7 km/s&8.4 km/s\\
Density of mirror ($\rho$) &2.2 g/cm$^3$&2.2 g/cm$^3$&2.3 g/cm$^3$\\
Mirror mass($M$)&40 kg&200 kg& 211 kg \\
Mirror diameter($r$)&340 mm&620 mm&620 mm\\
Mirror thickness($t$)&200 mm&300 mm&300 mm\\
Mirror temperature($T$)&300 K&300 K&10 K\\
\hline
\hline
\end{tabular}
\end{center}
\caption{Specification of Advanced LIGO
\cite{Ju2006a,Ju2006b,Zhao2005}, and ET-C
\cite{Hild2010a}.\label{tab:specification2}}
\end{table}

%Hild did not show the curvatures of mirrors. The curvature in Table
%\ref{tab:specification2} was derived from his parameters, cavity
%length and beam radii at mirrors. It is supposed that the
%curvature of front and end mirrors is the same. The beam radius
%$w_{\rm i}$ at mirror is described as
%\begin{eqnarray}
%{w_{\rm i}}^2 &=& \frac{L \lambda}{\pi |g_{\rm i}|}\sqrt{\frac{g_1 g_2}{1- g_1 g_2}}, \\
%g_{\rm i} &=& 1- \frac{L}{R_{\rm i}}.
%\end{eqnarray}
%Since $w_{\rm i}$ is 12 cm and $L$ is 10 km, the curvature of the
%mirrors is 5070 m.

%Hild supposed that the mass of a mirror $M$ is 120 kg. It implies
%that the aspect ratio (the ratio of the mirror thickness to
%diameter) is smaller than those of usual interferometric
%gravitational wave detectors because of the large beam radius (12
%cm). However, in order to simplify the discussion, it is assumed
%that the aspect ratio is the same as usual one (0.6 in LCGT). The
%mirror radius must be at least 2.5 times larger than the beam
%radius because the diffraction loss must be enough small
%\footnote{In this case, the diffraction loss is 3.7 ppm.}. Thus,
%the mirror radius of ET is 30 cm. The mirror radius of LCGT is
%12.5 cm. The volume of a mirror of ET is $(30/12.5)^3=2.4^3=14$
%times larger than that of LCGT. The mirror mass of LCGT is about
%30 kg. If the ET mirrors are made from sapphire as like LCGT, the
%mirror mass of ET is 410 kg. If the ET mirrors are made from
%silicon (densities of silicon and sapphire are 2.3 g/cm$^3$ and 4
%g/cm$^3$), the mirror mass is 230 kg.

\textbf{Parametric instability of Einstein Telescope: Maximum of R of Einstein Telescope}
\nopagebreak

The strength of instability $R$ is written as
\begin{equation}
R \propto \frac{4PQ_{\rm m}Q_{\rm o}}{McL{\omega_{\rm m}}^2}.\label{Rmax}
\end{equation}
Let us compare the maximum $R$ of ET
with that of Advanced LIGO. It is supposed that $Q_{\rm m}$ is the same. 
We must take the difference of power $P$ and mass $M$ into account. 
It must be noted that the ratio $Q_{\rm o}/L$ is proportional to the factor ${\cal F}/\lambda$.
The resonant frequency 
$\omega_{\rm m}$ is inversely propotional to the mirror size. 
It is assumed that scale of mirror is propotional to the third root of 
mirror volume (ratio of mass to density). 
We should consider the material difference in the case of ET-LF. 
The resonant frequency of mirror $\omega_{\rm m}$ is inversely propotional to the sound velocity.

The ratio of $R$ of ET-HF to that of Advanced LIGO is 
\begin{equation}
\left(\frac{3 {\rm MW}}{0.83 {\rm MW}}\right)\left(\frac{850}{1250}\right) \left(\frac{40 {\rm kg}}{200 {\rm kg}}\right) \left(\frac{200 {\rm kg}}{40 {\rm kg}}\right)^{2/3}=1.4.
\end{equation}
The ratio of $R$ of ET-LF to that of Advanced LIGO is 
\begin{equation}
\left(\frac{18 {\rm kW}}{0.83 {\rm MW}}\right)\left(\frac{850}{1250}\frac{1064~{\rm nm}}{1550~{\rm nm}}\right) \left(\frac{40 {\rm kg}}{211 {\rm kg}}\right) \left(\frac{211 {\rm kg}}{40 {\rm kg}}\frac{2.2 {\rm g/cm}^2}{2.3 {\rm g/cm}^3}\right)^{2/3} \left(\frac{5.7 {\rm km/s}}{8.4 {\rm km/s}}\right)^2=2.6\times10^{-3}. 
\end{equation}
The maximum of $R$ in Advanced LIGO is on the order of 100 \cite{Ju2006b}. Therefore, the parametric instability of ET-LF 
is not a problem. On the other hand, maximum $R$ of ET-HF is comparable with that of Advanced LIGO. We will discuss 
parametric instability of only ET-HF.

\textbf{Parametric instability of Einstein Telescope: Number of unstable modes of ET-C-HF}
\nopagebreak

The number of unstable modes are proportional to the product of elastic mode density and 
optical mode density.
The mode density of the elastic mode is proportional to the cubic
of the ratio of the mirror size to the sound velocity. Thus, 
the elastic mode density of ET-HF is 5 times larger than that of Advanced LIGO because
of the difference of mirror size. 

The ratio of the optical transverse mode interval to the free spectrum range is
described as
\begin{eqnarray}
&&\frac1{\pi}\cos^{-1}\sqrt{g_1 g_2},\\
&&g_n = 1-\frac{L}{R_n}.
\end{eqnarray}
There are 8 and 4 transverse optical modes in a free spectrum
range of Advanced LIGO and ET-HF interferometers. Since the cavity of ET is 2.5
times longer, the free spectrum range is 2.5 times smaller.
Therefore, the optical mode density of the ET-HF interferometer is 1.4 times larger.
The reason why the optical mode density of ET-HF is comparable with that of Advanced LIGO 
nevertheless the cavity length of ET-HF is longer stems from the difference of beam shape, 
TEM$_{00}$ and LG$_{33}$. Since the beam in Advanced LIGO is Gaussian (TEM$_{00}$), a larger beam radius 
implies smaller interval between optical transverse modes. On the other hand, ET-HF adopts 
the LG33 mode, higher optical transverse mode. The radius of higher mode is larger than that of TEM00. 
Thus, it is not necessary to increase the radius of fundamental mode (TEM$_{00}$). This effect cancels 
that of longer length of cavity. 
 
In total, the number of the instable modes, which is proportional
to the product of the elastic and optical mode densities, of the
ET-HF interferometer is 7 times larger than that of the Advanced LIGO interferometer. 
The main reason of the difference of Advanced LIGO and ET-HF is that of size of mirrors. 

\textbf{Parametric instability of Einstein Telescope: Mirror curvature dependence of ET-C-HF}
\nopagebreak

The instability strength $R$ is a function of the transverse optical
mode frequencies. How the curvature variation affects
the $n$-th optical transverse mode was calculated.
The result is 1.2$n$ Hz/m in ET-HF. This value is smaller than that of 
Advanced LIGO (15$n$ Hz/m). The longer baseline and adopting LG33 mode 
make mirror curvature dependence of instability smaller. 

\textbf{Parametric instability of Einstein Telescope: Comparison with ET-B}
\nopagebreak

Here parametric instability of ET-B \cite{HildETconventional}, which is sensitivity in old design, 
is compared with that of ET-C (the details about parametric instability of ET-B is 
discussed in Ref.~\cite{Yamamoto2009}). 
From point of view of parametric instability, ET-C is better than ET-B.
Obviously, the first reason is that ET-C-LF has no serious problem about this instability. 
The second reason is that ET-C-HF has (about 4 times) less instable modes and (about 3 times) 
weaker mirror curvature dependence.
This is because of the difference of beam shape; Gaussian beam (TEM00) in ET-B and LG33 in ET-C-HF. 

\textbf{Parametric instability of Einstein Telescope: Preliminary results of calculation for parametric instability of ET-C-HF using finite element method}
\nopagebreak

%{\bf The details will appear here.}

%{\bf I hope that the details of ET-C-HF can be discussed. Daniel Heinert and Sergey Strigin will calculate it.}

%{\bf 
The preliminary calculation for instability of ET-C-HF using finite element method 
is introduced here. At first, it must be noted that 
further investigation is necessary.
We consider all elastic modes below 30~kHz (twice times the free spectral range of cavity). The anti-Stokes modes are 
taken into account. Therefore, Eq.~(\ref{R}) should be rewritten as \cite{Kells2002}
\begin{equation}
R = \sum_{\rm optical\ mode} \frac{4PQ_{\rm m}Q_{\rm
o}}{McL{\omega_{\rm m}}^2} \frac{\Lambda_{\rm o}}{1+\Delta
\omega^2/{\delta_{\rm o}}^2}-\sum_{\rm optical\ mode(anti)} \frac{4PQ_{\rm m}Q_{\rm
{o(anti)}}}{McL{\omega_{\rm m}}^2} \frac{\Lambda_{\rm o(anti)}}{1+\Delta
\omega_{\rm anti}^2/{\delta_{\rm o(anti)}}^2}, \label{R_antiStokes}
\end{equation}
The number of unstable modes is only 5. One of the reasons is that some Stokes modes are canceled by 
anti-Stokes modes effectively. If the anti-Stokes modes are neglected, there are 17 unstable modes. 
In this calculation, the splitting of the degenerated elastic mode due to imperfectness of mirrors 
is not taken into account \cite{Strigin2008a,Strigin2008b}. 
Since this effect increases the number of unstable modes, it should be considered in a next step. 
%However, even if the anti-Stokes modes are neglected, the number of unstable modes is smaller than 
%Advanced LIGO (20$\sim$60). Previous discussion predicts that number of unstable modes is about 7 times 
%larger than that of Advenced LIGO. The reason should be investigated. One of the possible reasons is that 
%beam shape (LG33) is not taken into account in previous discussion, i.e.\ the complicate beam shape might 
%make $\Lambda_{\rm o}$ smaller (Daniel and Sergey are comparing the overlap factors for some modes of the
%ET-HF standard configuration. Some results?). 

We calculated the instability with the other mirror curvature. 
The curvature of one mirror is changed. The new curvature is 5743~m.
In this case, the number of unstable modes is 5. 
So, the number of unstable modes does not depend on mirror curvature 
supporting the conclusion of the previous discussion.

%Shape of optical and elastic modes in the worst case (highest $R$)?}

\textbf{Instability suppression in ET-C-HF}
\nopagebreak

Although the strength of the instability of the ET-HF interferometer
is comparable with that of Advanced LIGO, the number of
the unstable modes is 7 times larger. The three methods for the
instability suppression in Advanced LIGO are being studied
\cite{Ju2006b,Gras2006,Ju2009}. Let us consider whether these three
methods (thermal tuning method, feedback control, Q reduction of
elastic modes) are appropriate for ET (the tranquilizer cavity
\cite{Braginsky2002} is one of the other methods. However, this
is not introduced here).

%{\bf I think something should be added.}

\textbf{Instability suppression in ET-C-HF: Thermal tuning method}
\nopagebreak

In the thermal tuning method \cite{Ju2006b}, a part of the mirror is
heated for curvature control. Since $R$ depends on the curvature,
the suppression of $R$ should be possible by this manner. However,
this method is not useful in ET-HF because the mirror curvature dependence of 
parametric instability is 10 times smaller than that of Advanced LIGO. 

\textbf{Instability suppression in ET-C-HF: Feedback control}

It is possible to control the light or the mirror so that the
parametric instabilities would be actively suppressed \cite{Ju2006b,Ju2009}.
If the number of unstable modes is smaller, feedback control is
easier. However, these are more difficult (active) methods than Q
reduction (passive method) of the elastic modes, especially, if
there are many unstable modes. Therefore, it is good that 
almost all modes are suppressed by passive method and that 
only strong instable modes are supressed by active method. 

\textbf{Instability suppression in ET-C-HF: Q reduction of elastic modes}
\nopagebreak

%\begin{figure}[thbp]
%\centering
%\includegraphics[width=0.8\textwidth]{Sec_Optics/coatingPI.pdf}
%\caption{Loss on the barrel surface. Although this loss decreases the
%elastic Q-values of the mirror, $Q_{\rm m}$, it has only a small
%contribution to the thermal noise \cite{Levin1998,Yamamoto2006}. Thus,
%this loss suppresses the parametric instabilities without an
%increase of the thermal noise \cite{Gras2006,Gras2004}.}
%\label{Fig:opt_coatingPI}   
%\end{figure}


This is a useful method \cite{Gras2006} for ET. The value of
$R$ is proportional to the Q-value of the elastic mode, $Q_{\rm
m}$, as shown in Eq. (\ref{R}). The Q-values of fused silica 
are about $10^8$ \cite{Numata2004}. The
maximum $R$ of ET is comparable with that of Advanced LIGO, on the order of 100
hundreds at most \cite{Ju2006b}. If the Q-values of the ET-HF mirrors become $10^6$, 
almost all modes become stable. Since
the mechanical loss concentrated far from the beam spot has a
small contribution to the thermal noise \cite{Levin1998,Yamamoto2002},
we should be able to apply additional loss on a barrel surface 
%as in Fig. \ref{coatingPI}, 
%without sacrificing the thermal noise
\cite{Gras2006,Gras2004}. 
The thermal noise caused by loss on barrel surface is 
$4.3 \times 10^{-25}/\sqrt{\rm Hz}$ at 100~Hz (details are described 
in Appendix after the summary of parametric instability). This is comparable with ET-HF sensitivity. 
However, this estimation is not precious (in order to simplify the discussion, 
it is assumed that laser beam is Gaussian although it is LG33). We should investigate thermal noise 
introduced by loss on barrel surface. 
%I hope that Daniel Heinert and Sergey Strigin will provide useful supports.}

We are able to introduce the loss on the barrel surface by the
coating Ta$_2$O$_5$, which is a popular material for the
reflective coating of the mirrors. According to Ref.
\cite{Yamamoto2006}, the loss angle of the
SiO$_2$/Ta$_2$O$_5$ coating is $(4 \sim 6) \times 10^{-4}$ between
4 K and 300 K. Since the loss of this coating is dominated by that
of Ta$_2$O$_5$ \cite{Penn2003}, the loss angle of Ta$_2$O$_5$ is $(8
\sim 12) \times 10^{-4}$. If the barrel loss dominates the mirror
Q, it would be expressed as \cite{Yamamoto2002}
\begin{equation}
\frac1{Q_{\rm m}} \sim \frac{E_{{\rm Ta}_2{\rm O}_5}}{E_{\rm substrate}}
\frac{2 d}{R}\phi,
\label{1/Qm}
\end{equation}
where $E_{{\rm Ta}_2{\rm O}_5},E_{\rm substrate}, d, R, \phi$ are
the Young's moduli of Ta$_2$O$_5$ and the mirror substrate, the
thickness of the Ta$_2$O$_5$ layer, the mirror radius and the loss
angle of Ta$_2$O$_5$, respectively. These values are summarized in
Table \ref{tab:coatingspe} \cite{Hild2010a, Yamamoto2006}. In order
to arrive at Q-values of $Q_{\rm m} \sim 10^6$, the Ta$_2$O$_5$ coating
thickness $d$ must be 80 $\mu$m in the case of ET-HF. 
\begin{table}[h]
\begin{center}
%\begin{ruledtabular}
\begin{tabular}{ll}
\hline
\hline
Young's modulus of the Ta$_2$O$_5$ ($E_{{\rm Ta}_2{\rm O}_5}$)
&$1.4 \times 10^{11}$ Pa\\
Young's modulus of the fused silica &$7.2 \times 10^{10}$ Pa\\
Mirror radius ($R$)&31 cm\\
Loss angle of Ta$_2$O$_5$ ($\phi$)&$10^{-3}$\\
\hline
\hline
\end{tabular}
%\end{ruledtabular}
\end{center}
\caption{Specification of the coating
\cite{Hild2010a, Yamamoto2006}.\label{tab:coatingspe}}
\end{table}

Recently, another method to reduce Q-values of elastic mode (using electrostatic actuator) is proposed 
\cite{Miller2011}. 
It should be considered in near future. 

\etbox{i}{box:result}{Summary}
{The parametric instability of ET-C is discussed 
(it is expected that the result of ET-D is similar). 
%It is better than that of ET-B. 
The instability of ET-C-LF is not a serious issue because of the small light power. 
The maximum strength of instability of ET-C-HF is comparable to that of Advanced LIGO. 
The mirror curvature depepndence of ET-C-HF instability is about 10 times 
weaker than that of Advanced LIGO because of longer baseline and LG33 beam.
However, the instable mode number is 7 times larger (In order to evaluate a precise number, 
the calculation using finite element method is necessary. This calculation is in progress). 
This is mainly owing to the larger mirrors of ET.
We should investigate how to suppress. 
Feedback control and Q reduction of elastic modes are candidates.}

\textbf{Appendix A : Mirror scale dependence of thermal noise by barrel surface loss}
\nopagebreak

The ET mirror is larger than the Advanced LIGO mirror. We must consider the
size effect on the thermal noise by the loss on the barrel surface. In order to
simplify the discussion, it is supposed that the mirror and laser
beam of ET are similar to those of Advanced LIGO (although the beam of ET-HF is 
LG33, not Gaussian). In short, the difference
between ET and Advanced LIGO mirrors is only the scale. The ratio of beam
radius to mirror radius is the same.
%In order to simplify the discussion,
%it is supposed that the beam radius and thickness of loss layer on
%barrel surface is proportional to mirror radius and that the ratio
%of mirror thickness to mirror diameter is constant.
Under these assumptions, Q-values of mirrors are independent of the
mirror scale, $a$. According to Ref. \cite{Levin1998}, the amplitude
of the thermal noise $\sqrt{G_{\rm coating}}$ is described as
\begin{equation}
\sqrt{G_{\rm coating}} \propto \sqrt{W_{\rm diss}},
\label{Levin}
\end{equation}
where $W_{\rm diss}$ is the dissipated power when the pressure of
which the profile $p(r)$ is the same as the laser beam is applied
on the mirror flat surface. If the loss is the structure damping,
the dissipated power is proportional to the elastic energy in the
loss layer on the barrel surface;
\begin{equation}
W_{\rm diss} \propto \int_{\rm barrel\ surface} {\cal E} dS \times d,
\label{Wdiss}
\end{equation}
where ${\cal E}$ and $d$ are the elastic energy density and the thickness of
the loss layer. The problem is how $W_{\rm
diss}$ depends on the mirror scale $a$. Since the mirrors are
similar, $dS$ and $d$ are described as
\begin{eqnarray}
dS &\propto& a^2,\\
d &\propto& a.
\end{eqnarray}
The elastic energy density ${\cal E}$ is proportional to the
square of the strain tensor. The strain tensor is proportional to
the pressure on the flat surface $p$. This pressure is inverse
proportional to the square of the scale. In short, the elastic
energy density is written as
\begin{equation}
{\cal E} \propto p^2 \propto \frac1{a^4}.
\end{equation}
We are able to obtain the relationship between the thermal noise
and the mirror scale;
\begin{equation}
\sqrt{G_{\rm coating}} \propto \sqrt{\frac1{a^4} \times a^2 \times a} = \sqrt{\frac1{a}}.
\label{parametric instability size correction}
\end{equation}

Above discussion is based on the assumption that the ratio of the
beam radius to mirror radius is constant. If the mirror size is
fixed, the thermal noise by the barrel loss is independent of the beam size
\cite{Yamamoto2002}. Therefore, the thermal noise by the barrel
surface coating is inverse proportional to the square root of the
mirror scale.

The thermal noise caused by surface coating for instability supression is evaluated. The 
thermal noise by the holes with loss on the barrel surface is investigated in Ref. \cite{Gras2004}.
The size of their mirror is 150~mm in diameter and 60~mm in thickness. The mass is 2.3~kg. Since 
the mass of ET-HF is 200~kg, the size of ET-HF mirror is about $(200/2.33)^{1/3}\sim4.4$ times larger.
The beam radius of ET-HF is 72.5~mm. If the ratio of beam radius to mirror radius is the same, 
the beam radius of ET-HF corresponds to 1.6~cm in Ref. \cite{Gras2004}. Figure 4 of Ref. \cite{Gras2004} 
shows that mirror thermal noise (displacement) with 1.6~cm radius beam is about $10^{-20}~{\rm m/}\sqrt{\rm Hz}$ at 100~Hz. 
Since there are 4 mirrors and cavity length is 10~km, it corresponds to 
a strain noise amplitude of $2 \times 10^{-24}~/\sqrt{\rm Hz}$.
We should take correction of size, Eq.~(\ref{parametric instability size correction}), into account.  
The Q-values of mirror with this loss in Ref.~\cite{Gras2004} 
is $1/(10^{-2}\times5\times10^{-4})=2\times10^5$. Since we assume 
that Q-value is $10^{6}$, we should take this difference into account. In total,
\begin{equation}
2 \times 10^{-24}~/\sqrt{\rm Hz} \times \sqrt{\frac{1}{4.4}} \times \sqrt{\frac{2 \times 10^5}{10^6}} = 4.3 \times 10^{-25}~\sqrt{\rm Hz}
\end{equation}
This value is comparable with the ET-HF sensitivity. 
However, this evaluation is not precise. For example, we assumed that beam is Gaussian. 
However, ET-HF will adopt LG beam. These effects should be investigated. 
%I hope that Daniel Heinert and Sergey Strigin will provide useful supports.}




