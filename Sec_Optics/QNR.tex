\FloatBarrier
\subsection{Quantum noise reduction techniques}\label{sec:qnr}
%\emph{Author(s): \textbf{Helge Mueller-Ebhardt}, Keiko Kokeyama, Sergey Tarabrin} \\

In a laser-interferometric
gravitational-wave detector, there are different types of noise
sources, which are usually categorized into quantum noise sources
and classical noise sources (cf.\
Sec.~\ref{sec:optlayout}). In terms of the noise, the main
difference in the sensitivity of the different topologies comes
from the spectral distribution of the quantum noise, even though there
could also arise differences in the susceptibility to the
classical noise, due to the fact that there are e.g.\ a different
number of mirrors or different shapes of cavities. Therefore, the choice
of topology is mainly defined by the choice of quantum noise reduction
techniques used.

\textbf{Quantum noise} The so-called free-mass standard quantum limit
(SQL)~\cite{Braginsky1968,Braginsky1999} on high-precision
measurements is imposed by the Heisenberg uncertainty principle,
when it is applied to free-falling test masses. The spectral
representation of this quantity, which falls off with one over
frequency in amplitude and only depends on the test-mass weight
and arm length of the detector, has become a standard reference
for the quantum noise of interferometric gravitational-wave
detectors. With the help of this reference one is able to compare
the quantum noise of different topologies and configurations
having the same test-mass weight and arm length. The quantum noise
of a gravitation-wave detector consists of two parts: the quantum
measurement noise, i.e.\ the direct imprecision of the measurement,
and the quantum back-action noise. For interferometric
gravitational-wave detectors, the direct measurement process
consists of counting the number of photons by recording the photo
current of a photo diode. The photons of a coherent beam arrive
according to a Poissonian distribution. The photon counting error
represents the direct measurement noise, which is usually called
photon shot noise. The power of this noise source is inversely
proportional to the circulating laser power. The measurement
back action is clearly given by the laser light fluctuating
radiation pressure which imposes a force onto the mirror and
causes the radiation-pressure noise in the measurement output. The
power of this noise source is, in contrast to the photon shot noise,
directly proportional to the circulating laser power. Since the
suspended mirrors in a gravitational-wave detector can be
approximated as free falling test masses in the direction of the
incident laser beam---due to a very low eigenfrequency of the
pendulum created by the mirror's suspension---and since the two quantum
noise sources are uncorrelated, they result in the free mass
standard quantum limit. In this case, the radiation-pressure noise
dominates the spectral density of the quantum noise at lower
frequencies while the shot noise dominates at higher frequencies.
Therefore, one in general needs to compromise between high optical
power for a low shot noise and not too high optical power in order
to cope with the radiation-pressure noise.

\etbox{h}{hbox:xylo}{Xylophone configuration}
{Based on the requirements on the quantum noise sources (and technical ones, cf. Sec.~\ref{sec:optlayout}), we have chosen a xylophone configuration ~\cite{Hild2010a} as an optimal design of ET (cf. Sec.~\ref{sec:xylophone}). Each detector in a xylophone configuration is split into two interferometers, one optimized for low frequencies, operating at low light power and the other optimized for high frequencies operating at high light power. A xylophone configuration resolves two major problems:
\begin{itemize}
\item The simultaneous usage of high circulating light power for increasing the high-frequency sensitivity and cryogenic mirrors for decreasing thermal noise. In a xylophone configuration the low-frequency interferometer utilizes relatively low optical power which does not pose a problem of heating the cryogenic mirrors, while the mirrors at room temperature in the high-frequency interferometer allow use of much higher light power.
\item Simultaneous decrease of photon shot noise and radiation pressure noise. The sensitivity of the radiation pressure noise-dominated low-frequency interferometer benefits from low light power, while the sensitivity of the shot noise-dominated high-frequency interferometer benefits from the high light power.
\end{itemize}}
On the other hand, the SQL is actually not a real limitation on
the quantum noise strength of a gravitational-wave detector.
Several methods for overcoming the SQL, which are suitable for
laser interferometric gravitational-wave detectors, have been
proposed. They have different, and often very special,
requirements on the optical topology. The quantum-noise reduction
techniques can be divided into two main groups, where the
classification is not chosen in terms of the topology, but by the
technique of how the quantum noise is reduced: the first one is
based on the principle, that the goal of the gravitational-wave
detectors is not the measurement of the test-mass position, which
is a quantum variable and thus cannot be measured continuously
with a precision better than the SQL, but rather the detection of
a gravitational-wave strain as a signal, which can be treated as a
classical (tidal) force acting on the test mass
mirrors~\cite{Braginsky2003}. It was shown that by introducing
cross-correlation between the quantum measurement noise and the
quantum back-action noise, arbitrary high sensitivity (in terms of
the quantum noise) can be achieved~\cite{KLMTV}---assuming the
absence of optical losses. The correlations are actually used here
to quantum-mechanically cancel the back-action noise in the
measurement output. Thus, this method relies clearly on {\bf
noise-cancellation techniques}. The second group of methods is
based on the idea that the spectral distribution of the SQL itself
is not a fixed constant, but depends on the test object dynamics,
i.e.\ on the (mechanical) susceptibility of the test mass, which
relates the test-mass motion to all forces acting on it.
Therefore, the free-mass SQL can be beaten by using a more
responsive object and thus increasing its signal displacement---the harmonic
oscillator as an example has much stronger response
to near-resonance forces and therefore a better sensitivity than
the free-mass SQL around the resonance frequency. Therefore, the
sensitivity gain is obtained not by delicate cancelation of the
quantum noise, but by a classical {\bf signal amplification}. More
details about quantum-noise-reduction techniques are given in
Sec.~\ref{subsec:qndopt}. In that section we will also see that with
those topologies there are different detector options for the main
interferometer: the position meter, the optical spring
interferometer, the speed meter, the optical transducer.
Furthermore, all these main interferometer detectors can then be
additionally equipped with the input-squeezing technique (for
details cf.\ Sec.~\ref{subsec:QNRsqz})
and the variational readout technique (cf.
Sec.~\ref{subsec:qndopt}).

\FloatBarrier
\subsubsection{Review of quantum non-demolition topology options}
\label{subsec:qndopt}
%\emph{Author(s): \textbf{Helge Mueller-Ebhardt}, Keiko Kokeyama, Sergey Tarabrin}\\

Already the second generation of laser-interferometric
gravitational-wave detectors (such as the Advanced LIGO
detector~\cite{AdvancedLIGOReference2009} and the the Advanced VIRGO
detector~\cite{AdV2009}) are expected to be limited by the quantum
noise---the shot and the back-action noise---nearly within the full
detection band. Within the context of the third generation of
detectors this aspect becomes even more important, since there is
an enormous effort in increasing the quality of the technical
components of the interferometer---such as the mirror and beam
splitter materials as described in Sec.~\ref{sec:optcomps};
the suspension systems (see Sec.~\ref{sec:suspension_systems});
the stability of the laser source (Sec.~\ref{sec:injection})---in order to
decrease the strength of all the
technical noise sources by a large amount. Beside parameters such
as the circulating optical power and the test-mass weight, the
spectral distribution of the quantum noise mainly depends on the
topology of the detector, including the injection strategy at the
bright port as well as the detection strategy at the dark port of
the interferometer. Therefore, the choice of the topology and
configuration of the detector are severe for the design of future
gravitational-wave detectors. Especially, because parameters as the
optical power and the test-mass weight will always be limited due
to technical reasons, the design of future gravitational-wave
detectors requires quantum-noise-reduction techniques. As we
have seen in Sec.~\ref{sec:lshape}, there are different
topology options available which can all be fitted into an
L-shaped geometry. With the different topologies one can build up
different types of detectors, having specific quantum noise
features, as we have reviewed within this design study. Many of
them have actually great potential in reducing the quantum noise,
but there is a big discrepancy in terms of readiness: some are far
away from being ready to be implemented into gravitational-wave
detectors, others have been already demonstrated experimentally as
a proof of principle or have been even already implemented into
gravitational-wave detectors. %In the following we will report on
%the sensitivity of
We have investigated and reported the potential sensitivities of the following topology options (see details in Appendix~\ref{app:QNR}).
\begin{itemize}
\item \textbf{Optical-spring interferometer}. Optomechanical coupling in the arm cavities of a
Michelson interferometer with detuned signal-recycling can induce
a restoring force onto the differential motion of the arm-cavities
mirrors---the optical spring effect, which
can up-shift the mechanical resonance frequency into the detection
band. The sensitivity of the detector is enhanced around
an additional resonance, the optical spring resonance.
\item \textbf{Speed-meter interferometer}. A speed
meter is able to surpass the standard quantum limit broadbandly by
removing the (frequency-independent) radiation-pressure noise from
the measurement output. Speed-meters can be realized in different ways,
for instance by using a sloshing cavity or polarizing optics in a Michelson interferometer,
using the Sagnac interferometer with the ring cavities, etc.
\item \textbf{Optical-inertia interferometer}. A
detuned signal-recycling cavity turns a speed meter interferometer
into an optical inertia
interferometer: the optomechanical
coupling influences the dynamics of the test-masses---it modifies
their dynamical mass by introducing an optical inertia. The hope is that in this way one
can create a test object which has a high resonance-type
mechanical susceptibility in a broad frequency band.
\item \textbf{Optical transducer with local readout}. The idea of such schemes is not to measure the
phase shift of the laser field via monitoring the outgoing
modulations fields at the dark port of an interferometer but to
measure the redistribution of optical energy directly inside an
interferometer by converting the gravitational-wave strain via
radiation pressure into real mirror motion. This motion should
then be sensed by an additional highly sensitive local meter.
\item (\textbf{Frequency-dependent) input-squeezing interferometer}. The squeezed field
injection with frequency dependent squeezing angle allows an
overall quantum noise reduction including the radiation pressure
noise (details in Sec.~\ref{subsec:QNRsqz}). Frequency-dependent squeezing can be achieved with special filter cavities.
The filters allow the preparation of squeezed states providing a frequency-dependent
squeezed quadrature which is adapted to the interferometers
quadrature rotation (details in Sec.~\ref{sec:filtercavities}). The injection of such a prepared squeezed
state leads to a quantum noise reduction over the entire detection
band.
\item \textbf{Variational-output interferometer}. Interferometer detectors can be equipped
with a balanced homodyne detection and with the
variational-output technique. Here filter
cavities are used to make the quadrature angle of the detected
output field frequency dependent and therefore realize a broadband
evasion of the radiation-pressure noise.
\end{itemize}
Our analysis shows that one of these techniques alone is probably
not able to reduce the quantum noise in the required broadband
way, but certain combinations among these techniques are possible.
Due to the technical readiness issues of the reviewed topology options, we propose to split the detector into
two dual-recycled, squeezed light-injected Michelson/Fabry-Perot interferometers, the low- and
high-power ones, as a conceptual design of the
3rd generation GW detector for the means of reduction of quantum noise (see
details about xylophone configuration in Sec.~\ref{sec:xylophone}). However,
the study of speed-meter topology of the low-power interferometer as a potential method of improving the low-frequency sensitivity should be included into the R\&D phase.

\subsubsection{Alternative topologies and interferometry types}\label{sec:alttop}
%\emph{Author(s): Sergey Tarabrin}\\
We investigated the possibility of implementation of displacement
noise reduction techniques in ET (see Appendix~\ref{app:DFI}). The
class of so-called displacement noise free interferometers allows
complete cancelation of all the information about the displacement
noises from the output signal of the interferometer at the cost of
significant reduction of the GW susceptibility. This is possible due
to the distributed nature of the GWs as opposed to the localized
nature of displacement noises. Another class of schemes allow partial
displacement noise cancelation of some of the optical elements (cavity
mirrors, beamsplitters, etc) in the linear combination of the
interferometer's output signals. However, both types of noise
reduction schemes suffer from either very weak GW susceptibility or
very impractical requirements (rigid platforms, etc) and uncanceled
noises (like laser noise) thus making them barely advantageous over
the conventional topologies.


We also investigated the alternative to laser interferometry---atomic
interferometry (see Appendix~\ref{app:atom}). Light pulse atom
interferometry can be thought of as a comparison of time kept by
internal atom clocks and optical wave of the laser. The incoming
gravitational wave changes the rate of time which can be seen in an
interferometer phase shift. The major advantage of the atom-light
interferometry over conventional optical interferometry is that the
atoms, playing the role of inertial sensors, are not subjected to the
external fluctuations in comparison with the mirrors, and thus do not
require sophisticated vibration isolation techniques. Although light
pulse atom interferometry has already found applications in atomic
clocks, metrology, gyroscopes, gradiometers and gravimeters, its
implementation in gravitational-wave detection requires detailed and
comprehensive study and further development of the noise-lowering
techniques. With the current available technologies atomic
interferometers cannot provide the same level of sensitivity as the
well-developed optical interferometers.

