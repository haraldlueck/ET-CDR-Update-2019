\FloatBarrier
\subsection{Alternative to laser interferometry: atomic
sensor}\label{app:atom} 
%\emph{Author(s): Sergey Tarabrin.}

It is in principle also possible to utilize light pulse atom
interferometry to detect gravitational waves. Light pulse atom
interferometry can be thought of as a comparison of time kept by
internal atom clocks and optical wave of the laser. The incoming
gravitational wave changes the rate of time which can be seen in
an interferometer phase shift. The major advantage of the
atom-light interferometry over conventional optical interferometry
is that the atoms, playing the role of inertial sensors, are not
subjected to the external fluctuations in comparison with the
mirrors, and thus do not require sophisticated vibration isolation
techniques. A phase shift measurement in an atomic interferometer
consists of three steps~\cite{2008_atomic}: first, the atomic
cloud is prepared, cooled to sub-microkelvin temperatures and then
launched. Atoms in the cloud are in the $|\textrm{ground}\rangle$
state and are freely falling after the launch. In the second phase
light pulses are applied. The ``beamsplitter'' $\pi/2$-pulse
places atoms in the superposition of two states:
$1/\sqrt{2}|\textrm{ground}\rangle+1/\sqrt{2}|\textrm{excited}\rangle$.
Since the internal state of the atom is correlated with its
momentum, atoms in ground and excited states acquire different
velocities, and thus both states become temporarily and spatially
separated. After some time the ``mirror'' $\pi$-pulse exchanges
the two components of the superposition:
$|\textrm{ground}\rangle\rightarrow |\textrm{excited}\rangle$,
$|\textrm{excited}\rangle\rightarrow|\textrm{ground}\rangle$, so
that the atoms will finally overlap. Finally, the second
``beam-splitter'' $\pi/2$-pulse makes the two branches of the atom
wave function interfere, in full similarity to a Mach-Zehnder
interferometer. The third phase of interferometry is detection.
The interference pattern can be extracted by measuring the
population of atoms in a given state, for instance, in the excited
state. The measured phase shift results from both the free-fall
evolution of the quantum state along each path in interferometer
and from the local phase of the laser which is imprinted on the
atoms at the moments laser pulses are applied. Since laser sources
and atomic interferometer can be separated by a significant
spatial distance, the incoming gravitational wave modulates the
latter, thus causing the modulation of the arrival time of the
laser pulses which enters the measured atomic phase shift. A
terrestrial based gravitational-wave detection with light-atom
interferometry can be realized in a vertical shaft with the linear
scale of $\sim 1\,$km. Two atomic interferometers of the linear
scale of $\sim 10\,$m are placed on the top and the bottom of the
shaft and are operated by common lasers. With a reasonable
measurement repetition time, a ground-based setup will have its peak
susceptibility to the gravitational waves around $\sim 1\,$Hz, which is
very interesting form the astrophysical point of view. Such a
setup allows performing differential measurements between two
atomic interferometers, which significantly suppresses the vibrational
and optical noises of the lasers. The vibration of the optical trap
which leads to different launch velocities is of less importance,
since the initial ``beam-splitter'' pulse is applied after the
atoms are launched. However, spread in velocities will enter the
measurement error through the gravity-gradients, since in the
nonuniform gravitational field atoms moving along different
trajectories experience different gravitational forces.
Gravity-gradients seem to be one of the major limiting factors
towards the increase of the sensitivity. Other noises sources come
from the variations of the magnetic field which change the atoms
energy levels, coupling of the Earth rotation to the fluctuating
transversal velocity of the optical trap. One of the dominating
noise sources with the technique currently available is the atomic
shot noise. It can be lowered by implementation of the sources
with more intense atom fluxes and/or preparation of the atoms in
squeezed states. Although light pulse atom interferometry has
already found applications in atomic clocks, metrology,
gyroscopes, gradiometers and gravimeters, its implementation in
gravitational-wave detection requires detailed and comprehensive
study and further development of the noise-lowering techniques.
With the current available technologies atomic interferometers
cannot provide the same level of sensitivity as the well-developed
optical interferometers. 