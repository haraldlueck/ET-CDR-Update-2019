\subsection{Displacement-noise-free interferometry}\label{app:DFI}
%\emph{Author(s): Sergey Tarabrin.} 
Most of the dominant noise sources in laser
interferometric gravitational-wave detection can be related to the
class of displacement noise: seismic noise, gravity-gradient
noise, various thermal noise sources, even the quantum back-action
noise. Each method of suppression or elimination of displacement
noise is usually suited for control of only one kind of noise:
seismic isolation, measurement and partial cancellation of gravity
gradients, cryogenics, quantum-noise-reduction schemes.
Displacement-noise-free interferometry (DFI) is the method of
displacement noise cancellation which aims at simultaneous
elimination of the information about all position fluctuations of
the test masses, but leaving a certain amount of information about
gravitational waves. All known DFI schemes can be divided into two
categories: schemes with complete and partial displacement noise
cancellation. Complete displacement noise cancellation relies on
the distributed nature of gravitational waves. While displacement
noise imprints on the optical phase only at the moments of the
laser beam reflection at the test masses (localized effect),
gravitational waves affect the laser beam along its optical path
(distributed effect). From the viewpoint of some local observer
the interaction of the gravitational wave with the interferometer
adds up to two effects~\cite{2005_local_observ}: the motion of the
test masses in the gravitational-wave tidal force-field (which is
indistinguishable from the action of fluctuating forces, therefore
it is a localized effect) and the direct coupling between the
gravitational wave and light (distributed red-shift effect). DFI
implies the cancellation of displacement noises along with the
localized part of the gravitational-wave effect, leaving the
distributed red-shift effect in the interferometer response. Since
the latter one has the order of $O[h(L/\lambda_{\textrm{GW}})^2]$
(where $h$ is the gravitational-wave amplitude, $L$ is the
interferometer linear scale and $\lambda_{\textrm{GW}}$ is the
gravitational wavelength), DFI has much weaker gravitational-wave
susceptibility than conventional gravitational-wave detectors in
the long-wavelength regime $L\ll\lambda_{\textrm{GW}}$. Complete
displacement noise cancellation can be achieved in an
interferometer with large enough number of test masses by
properly combining several response signals~\cite{2006_DTNF_GW_detection}. For instance, 2- and 3-dimensional
setups composed of two Mach-Zehnder interferometer topologies
sharing the beam-splitters~\cite{2006_interferometers_DNF_GW_detection}. The
gravitational-wave response of the 2-dimensional scheme has the
order of $O[h(L/\lambda_{\textrm{GW}})^3]$, while the one of the
3-dimensional scheme is of the order of
$O[h(L/\lambda_{\textrm{GW}})^2]$. For comparison, the
gravitational-wave response of the conventional Michelson
interferometer is $O[h(L/\lambda_{\textrm{GW}})^0]$.
Implementation of time-delay devices, while improving the
strength of the gravitational-wave response, limits the
sensitivity by adding noise~\cite{2007_time_delay}. Another
class of DFI schemes with partial displacement noise cancellation
aims on keeping strong enough gravitational-wave susceptibility,
with either $O[h(L/\lambda_{\textrm{GW}})^0]$ or
$O[h(L/\lambda_{\textrm{GW}})^1]$ leading orders in the response.
This can be achieved with linear Fabry-Perot cavities, ring
cavities, etc. A single Fabry-Perot cavity, double-pumped through
both mirrors, allows elimination of their displacement noise in
the proper linear combination of the responses, however, the
sensitivity remains limited due to laser noise and displacement
noise of all the auxiliary optics~\cite{2008_DFI_FP_toy_model}.
Modification of this scheme with two cavities placed symmetrically
allows complete displacement noise cancellation but does not allow
laser noise cancellation~\cite{2008_DFI_2FP}. In a symmetric
double Michelson interferometer with the arm-cavities the
sensitivity is limited by the noise of the local oscillators used
for detection of the transmitted waves in the arms; in addition,
this scheme requires placing several mirrors rigidly on a single common
platform, which is very impractical~\cite{2009_DFI_MFP}.
Double pumping of the resonant ring cavity allows cancellation of
its mirrors noise and laser noise, but cannot deal with
displacement noise of the mirrors and beam-splitters used to
produce the pumping waves~\cite{2008_resonant_speedmeter_DFI}. To
summarize, all detectors with either complete or partial
displacement noise cancellation consist of combinations of many
different topologies and therefore differ in general significantly
from the conventional detectors especially in terms of the
complexity and either have very weak gravitational-wave
susceptibility or impractical requirements (like rigid platforms)
to operate, or suffer from uncancelled noises, thus making them
hardly advantageous over the conventional non-DFI topologies.
