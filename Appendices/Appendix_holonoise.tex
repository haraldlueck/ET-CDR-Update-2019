\subsection{Holographic noise}\label{app:holo}
%\emph{Author(s): Sergey Tarabrin.} 
It is
currently widely assumed that the holographic principle, developed
by G.~'t~Hooft~\cite{Hooft1993} and L.~Susskind~\cite{Susskind1995}, should be the fundamental
constituent part of any unified theory of quantum gravity.
It says
that that the physical theory defined in the space-time of
dimensionality $D$ is equivalent to another theory defined on the
boundary of dimensionality $D-1$. The most known mathematical
realization of holographic principle is the AdS/CFT correspondence
by J.~Maldacena: string theory in anti-de Sitter space-time is
equivalent to conformal field theory on its boundary~\cite{Maldacena1997}.
It follows
from the holographic principle that if the volumetric system can
be described by the theory on the boundary, then the maximal
number of volumetric degrees of freedom should not exceed the
number of their ``images'' on the boundary. Since the
``classical'' field-theoretical informational content of the
region of space is defined by its volume, such a description
contains much more degrees of freedom than allowed by the
holographic entropy bound. Therefore, our 3-dimensional world must
be ``blurry'' in order to match the number of degrees of freedom
inscribed on some 2-dimensional holographic surfaces. The
holographic uncertainty is a particular (highly speculative)
hypothesis proposed by C. Hogan about how the holographic
principle works in a flat space-time~\cite{2009_holonoise}. He
posited that in order to preserve the holographic nature of space
time, it must have diffractive nature described by the wave
functions of transversal position distribution of matter-energy~\cite{2009_m-theory}, i.e.\ the transversal coordinates of two
particles (test masses) separated by a distance $L$ in a
longitudinal direction should no longer commute:
$[\hat{x}_1,\hat{x}_2]=il_\textrm{P}L$, where the commutator is
defined on the light-like geodesics only. The corresponding
uncertainty relation reads $\Delta x_1\Delta x_2\geq
l_\textrm{P}L$ meaning that the relative transversal positions of
two test masses cannot be measured with infinite precision. The
holographic uncertainty relation implies that the measurement of
the transversal position of a single test mass with the optical
signals will yield uncertain results with $\Delta
x\geq\sqrt{l_\textrm{P}L}$, where $L$ stands for the distance the
light wave travels between the two measurements. This holographic
fuzziness with associated uncertainty $\Delta x$ should be seen in
precise interferometry, otherwise it would be possible to
distinguish more test-mass configurations than is allowed by the
holographic entropy bound. Thus, in a Michelson interferometer
the measurement of the beam-splitter transversal position relative to
the direction of the incident laser beam should yield uncertain
results. Uncertain measurement results produce fluctuating
time series, i.e.\ the noise called holographic noise. The minimal
level of expected holographic noise corresponds to the Gaussian
space-time wave functions which minimizes the holographic
uncertainty relation, much like in usual quantum mechanics. In the
frequency region $f\ll c/L$ holographic noise power spectral
density is frequency-independent and equals to
$S(f)=2t_\textrm{P}L^2/\pi$, or effective metric strain
$h(f)=\sqrt{S(f)/L^2}=\sqrt{2t_\textrm{P}/\pi}=1.84\times10^{-22}/\sqrt{\textrm{
Hz}}$
with $t_\textrm{P}$ standing for Planck time. Holographic noise
prediction is thus fixed with no free parameters, therefore the
hypothesis can be either confirmed or ruled out experimentally.
Holographic noise signatures are currently being looked for in the
noise spectrum of GEO-600 interferometer. However, the available
sensitivity does not allow to make unambiguous conclusions. Since
the space-time wave function universally defines the transversal
distribution of mass-energy, holographic noise should exhibit
particular cross-correlation features. Namely, two closely
positioned interferometers should produce correlated measurements
of the holographic displacement, because they occupy nearly the
same space-time volume and thus holographic motion of their test
masses (beam-splitter, in particular) is defined by nearly the
same wave function. If the two interferometers are aligned along
their arms and are displaced by $\Delta L\ll L$ along one of them,
then the cross-correlation spectral density equals to
$S(f)=2t_\textrm{P}L^2[1-(\Delta L/L)]/\pi$. This expected feature
of the holographic noise is to be tested in the Fermilab holometer
which is currently under construction~\cite{2010_holo_proposal}.
For a Michelson interferometer with cavities in the arms, the
effective metric strain equals to
$h(f)=N^{-1}\sqrt{S(f)/L^2}=N^{-1}\sqrt{2t_\textrm{P}/\pi}=N^{-1}1.84\times10^{
-22}/\sqrt{\textrm{Hz}}$,
where $N$ is the average number of photon round trips inside the
cavities. The reason for the $N^{-1}$ factor is that the cavities
effectively lengthen the arms for the gravitational waves (this
holds true for the frequencies $f\lesssim c/2LN$), thus amplifying
the response to the gravitational waves, but do not change the
beam-splitter holographic displacement spectrum~\cite{2009_holonoise}. With planned transmittances of the
arm-cavities input mirror and the end mirror of $7000$ and $10$~ppm, respectively, the number of photon round-trips inside the ET
cavities equals to $N\approx 277$, thus lowering the holographic
metric strain to
$h(f)\approx0.66\times10^{-24}/\sqrt{\textrm{Hz}}$. The
development of sound theoretical models and experimental test are
under way. If it turns out that the holographic noise is a serious
issue for the Einstein telescope gravitational-wave detector, the
optical design (e.g.\ cavity finesse) has to be adapted. 