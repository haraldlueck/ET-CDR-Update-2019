\FloatBarrier
\subsection{Description}
Interferometric gravitational wave detectors are large and complex devices and the selection of their site is an issue of great importance. The selected site should allow the highest possible level of scientific productivity at reasonable cost of construction and operation, and at minimal risk. Of paramount importance are the selection criteria that impact the scientific potential of the observatory. These include natural and anthropogenically generated seismicity and site geological constraints that affect critical parameters such as interferometer arm lengths. The first section provides the requirements for the site and infrastructure of Einstein Telescope. An important aspect within these requirements is the allowed seismic displacement noise, which is addressed according to source frequency. Seismic sources include the ambient seismic background, microseismic noise, meteorologically generated seismic noise, and cultural seismicity from anthropogenic activity. 

The second section describes the Newtonian noise which constitutes one of the fundamental infrastructure limitations. Newtonian noise originates from fluctuations in the surrounding geological and atmospheric densities, causing a variation in the gravitational field. Results from new analytical formalisms and finite element models are presented for subterranean detectors. Estimates for Newtonian noise are derived for different geologies. Starting from these models we show that it is possible to deploy seismic sensor arrays that monitor seismic displacements and filter the detector data with Wiener or Kalman filters.

The third section discusses the site selection. As part of the site selection and infrastructure program, a total of 11 European sites were systematically characterised to catalogue regions within Europe that would comply with the site demands. The data that were logged include the local seismic activity, geology, availability of existing infra-structure, population density and local construction costs. 

Finally, we discuss the subterranean infrastructure and cost aspects for caverns, shafts, tunnels, vacuum, cryogenic and safety systems for the Einstein Telescope observatory.
\FloatBarrier
\subsection{Executive Summary}
The design study working group on site selection and infrastructure has carried out detailed site studies at 15 locations in 11 different countries. Furthermore, a comprehensive Infrastructure Reference Design of an underground detector was realised. The site study has revealed several promising EU underground sites that comply with the low seismic background performance requirements. In order to ascertain that all site characterisation procedures were according to the highest standards, measurements and data collection were carried out in collaboration with the Observatories and Research Facilities for European Seismology (ORFEUS) which is maintained by the seismology department of the Royal Dutch Meteorological Institute (KNMI). Site specific issues were discussed with geologists and representative from established underground laboratories (LNGS, LSM, Canfranc, HADES, Dusel and Kamioka) and mines (in Finland, Germany, Hungary, Italy and Romania). In addition, there was close contact with particle physics initiatives ($e.g.$ CERN, DESY and ILC).

The infrastructure definition of the reference design contains surface buildings, tunnels, caverns, shafts, vacuum envelope,  cryogenic infrastructure and safety systems.  With the choice of combining a triple triangular detector topology with a xylophone detector design, Einstein Telescope amalgamates an optimised planned disbursement for a staged construction with a realistic proposal for a robust, highly sensitive, wide-band gravitational wave observatory. Through the design process, the vacuum system, caverns, and tunnel diameter are optimised such that the total infrastructure can be used for decades after its construction. The excavation of underground tunnels, caverns, and shafts will occur over a period of approximately four years. The observatory will be built in stages; the first construction stage containing a single 10\,km xylophone detector. Later the second stage will incorporate the second and third 10\,km xylophone detectors.



