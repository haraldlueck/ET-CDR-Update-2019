\FloatBarrier
\subsection{Cost estimate for the Intrastructure Reference Design}

Preliminary cost analyses have been performed in order to optimize the cost-to-performance ratio for the Infrastructure Reference Design for Einstein Telescope. Various component costs were estimated, options compared and significant cost-driven changes were implemented will maintaining the scientific performance goals. The estimates were based in some cases on European-wide tenders and by using the lowest reasonable price for the dedicated specifications. In addition, estimates were made of the explicit work force needed to support the respective activity.

The total estimated value for the Infrastructure Reference Design is 775 MEuro (in 2011 \euro). The cost is determined for a single xylophone detector mode. This configuration constitutes the phase I instrumentation of the project. Phase II will be realized later and features a triple xylophone interferometer configuration.

The site specific costs are related to the direct costs for providing the infrastructure to site the observatory and are estimated at 592 MEuro (2011 value level). These costs include the underground civil facilities, services (electricity and water distribution), buildings and surface construction. The actual site-specific costs will depend on the location where the observatory is constructed, and the facilities that already exist at that location. For the estimates we have used 260 \euro $/ \mathrm{m^3}$ for the excavation costs. This includes digging cost, soil transportation and deposit, and finishing the floor and cavern walls with concrete. In addition, construction machines, running cost, maintenance, preparation and management costs are included. Several underground sites in Europe have been identified that comply with such costs, although a significant bandwidth applies. The number is also compliant with the site costs for LCGT in Japan.

The infrastructure costs include the direct costs for the safety and security systems, cooling and ventilation systems, the vacuum system and the cryogenic infrastructure. These costs are based on both the experience at GEO, Virgo and worldwide tenders. In order to realize the vacuum system at the estimated cost, it will be needed to set-up three vessel construction facilities at the observatory site.
 



