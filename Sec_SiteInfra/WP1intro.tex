\subsection{Description}
%
% put here the description of the content of the section
%
\emph{
Responsible:  WP1 coordinator, J.v.d.Brand  \\
}
Interferometric gravitational wave detectors are large and complex and the selection of their site is an issue of great importance. The selected site should allow the highest possible level of scientific productivity at reasonable cost of construction and operation, and at minimal risk. Of paramount importance are the selection criteria that impact the scientific potential of ET. These include natural and anthropogenically generated seismicity and site geological constrains that affect critical parameters such as interferometer arm lengths. The first section of this chapter provides the requirements for the site and infrastructure of Einstein Telescope. An important aspect within these requirements is the allowed seismic motion, which is addressed according to source frequency. Seismic sources include the ambient seismic background, microseisms, meteorologically generated seismic noise, and cultural seismicity from anthropogenic activity. 

The second section describes the background on one of the fundamental infrastructure limitations called Newtonian noise. Newtonian noise originates from fluctuations in the surround geologic and atmospheric density, causing a variation in the Newtonian gravitational field. New analytical formalisms and finite element models are presented for subterranean detectors giving an estimate for NN in ET for different geologies. Using these models we show that it is in principle possible to deploy seismic sensor arrays that monitor seismic displacements and filter the detector data using Wiener or Kalman filters.

The third section discusses the ET site selection. As part of the site selection and infra-structure program for ET, 11 European sites were systemically characterised to catalogue regions within Europe that would comply with the ET site demands. Among the data that were logged were the local seismic activity, already existing local infra-structure, and population density. 

Finally, we discuss the subterranean infrastructure and cost aspects for caverns, tunnels, vacuum, and cryogenics for the ET site.
\FloatBarrier
\subsection{Executive Summary}
%
% put here a summary of the main achievement/statements of this section
%
\emph{
Responsible:  WP1 coordinator, J.v.d.Brand  \\
}
%Put here a summary of the main achievement/statements of this section \dots \par
Throughout the ET design study, the site selection and infrastructure design working group has seen an opulent evolution from a site study at 11 locations in 9 different countries, to a completely designed underground detector infrastructure. The site study has revealed several promising underground sites that comply with ET low seismic background performance requirements. In order to ascertain all site characterisation procedures were carried according to seismologic standards, measurements and data collection was carried out in collaboration with the Observatories and Research Facilities for European Seismology (ORFEUS) which is maintained by seismology department of the Dutch bureau of Meteorology. 

%At any site, seismic noise causes a secondary noise which cannot be shielded from. The local seismicity causes perturbations in the local gravity field, which are predicted by the so-called Newtonian or gravity gradient noise in gravitational wave detectors. Whereas the sensitivity of currently operating gravitational wave detectors is not limited by this noise source, second and third generation detectors will be sensitive to these gravity gradients at 10\,Hz and below. In order to predict the effect of gravity gradient noise in subterranean settings, a new analytical formalism has been developed. The analytical results were verified by finite element analysis allowing model expansion by incorporating complex geologies and inhomogeneous soils. Finally, using the analytical model and finite element results, the knowledge of the local seismicity was used to create optimal filters to remove (with certain efficiency) the gravity gradient noise from the gravitational wave detector data. 

The infrastructure definition contained several aspects such as the vacuum envelope, cryogenic infrastructure, tunnels, and caverns that are transverse through the detector optical configuration and suspension working groups. With the choice of combining a triple triangular detector with a xylophone detector topology, ET amalgamates an optimised planned disbursement for construction with a realistic proposal for a robust, highly sensitive, wide-band gravitational wave observatory. Through the design process, the vacuum system, caverns, and tunnel diameter are optimised such that the total infrastructure can be used for decades after its construction. The excavation of underground tunnels, caverns, and halls will occur over a period of approximately two years, where the ET observatory will be built in stages; the first construction stage containing a single 10\,km xylophone detector. Later the second stage will incorporate the second and third 10\,km xylophone detector.



