On actual engineering surfaces it is reasonable, as a first approximation \cite{sparrow_1965, siegel-howell-3}, to represent the reflectance
$\rho$ as being divided into diffuse $\delta$ and mirror specular $\mu$
components:
\begin{equation}
\label{eq:reflectivity}
\rho = \delta + \mu
\end{equation}
In addition,
we assume an opaque gray body, whose radiation
is emitted diffusely according to Lambert's cosine law
\cite{siegel-howell-3}. The emitted flux depends on
the absolute temperature, $T$, of the surface, the surface
emissivity, $\epsilon$, and the Stefan-Boltzmann constant, $\sigma$, in the
combination $\epsilon\sigma T^4$. Kirchoff's law states that the same
surface element will absorb only the fraction $\epsilon$ of the incident
radiation, while reflecting the fraction $\rho = 1 - \epsilon$, so
that
\begin{equation}
\label{eq:sum}
\epsilon + \delta + \mu = 1
\end{equation}
In the diffuse limit ($\mu =0$, $\rho=\delta$), we let $G$ represent the radiant flux incident on a unit
surface. Then for a diffusely emitting surface with a
diffuse reflectance, $\delta$, the radiosity, $J$, given
by
\begin{equation}
\label{eq:radiosity}
J = \epsilon\sigma T^4 + \delta G
\end{equation}
represents the diffusely distributed radiant flux leaving
a unit surface element \cite{tsai_1986}. 
The net inward radiative heat flux, $q$, is then given the difference between the irradiation and the radiosity:
\begin{equation}
\label{eq:influx}
q = G - J
\end{equation}
Using eq.~\ref{eq:radiosity} and eq.~\ref{eq:influx} we can eliminate J and obtain a general expression for the net inward heat flux into the opaque body based on $G$ and $T$. From eq.~\ref{eq:sum} with $\mu = 0$ we get $\epsilon = 1 -\delta$, thus $q$ is given by:
\begin{equation}
\label{eq:qgray}
q = \epsilon\left( G - \sigma T^4 \right)
\end{equation}
Starting from Eq.~\ref{eq:qgray} the contribution to the thermal radiation heat transfer coming from diffuse reflectivity was calculated for the geometry given in Fig.~\ref{fig:vacuum-tube} by a finite-element model.
