
\FloatBarrier
\subsection{Technologies to be developed}
\subsubsection{Introduction}
The optimal site for a European third generation gravitational wave observatory should ideally provide the following characteristics \cite{GGCellaCuoco}: 
\begin{itemize}
\item{} The level of seismic and gravity gradient noise should be compatible with the anticipated sensitivity requirements for Einstein Telescope. 
\item{} The local geology should be able to support the required infrastructure and potentially allowing the application of gravity gradient subtraction techniques. 
\item{} An active involvement of a team of local scientists as well as the funding body of the corresponding country. 
\end{itemize}

Research and development is needed to achieve the sensitivity foreseen for the third generation of underground and cryogenic interferometric detectors. The fundamental noise sources that limit the design sensitivity at very low frequency (1 - 10 Hz) are seismic displacement noise and gravity gradient noise (GGN). While the first can be handled by advanced superattenuators, the latter affects the detector sensitivity since the vibrating soil can directly couple to the suspended optics. Furthermore, seismic noise complicates the control of the seismic filter chain (so-called control noise).

The R\&D program includes studies at candidate sites, the development of suitable sensor networks, and techniques to quantify and understand the low frequency noise contributions. The studies encompass the development of GGN subtraction schemes. 

Initial steps along these directions have been taken in de Einstein Telescope design study \cite{ET}. A first round of seismic measurements at various European underground sites has been completed.
In addition, analytical calculations were carried out to address GGN. These calculations assume infinite and completely uniform bedrock around the caverns. They do not take into account the effects of the surface above, and consider uniform 
and frequency-independent seismic excitation. In an attempt to go beyond these limitations, numerical, finite-element modeling of the site, including effects of stratification, seismic noise reduction as a function of depth, varying speed of sound, varying coherence levels and the effects of the surface above, has been implemented to achieve more accurate GNN reduction estimates. At present, these models are not yet supported by suitable experimental measurements that must provide all the necessary preliminary information to make these evaluations reliable.

The R\&D program envisions extensive field studies of candidate sites for the Einstein Telescope observatory. These studies will be performed by local scientific teams with arrays of high quality commercial seismic sensors. In addition, the development and integration of prototypes of both unidirectional opto-mechanical monolithic and MEMS-based sensors is foreseen. These sensors will be optimized to be used standalone for both seismic noise monitoring and geophysics applications, with particular attention in their design to the necessary immunity to environmental noise. The arrays will be upgraded as soon as new improved sensors are available.
Data from these arrays will provide information about attenuation of seismic noise with depth, dependence on geology, $e.g.$ sediments, hardrock and salt, and coherence length of signals between various sensors. We will attempt to decompose seismic noise according to surface waves, such as Rayleigh and Love waves, and body waves. Furthermore, daily variations will be exploited in order to estimate contributions from cultural noise. Although limited seismic information is available from a variety of sites, the proposed studies will include long-term observations. These studies will aid to identify a possible R\&D path towards seismic isolation systems, while the data will be used as input to the underground-surface site decision making process. It will allow identifying time-dependent contributions in Newtonian background from surface seismic compression waves, weather activity (variations in atmospheric pressure), ground-water dynamics, slow gravity drifts of geophysical origin, cultural noise (moving object: humans, machines, $etc.$). The proposed studies will quantify these various contributions and identify approaches to limiting gravity gradient noise with active correction systems with data from seismic sensors, accelerometers, strainmeters, rock thermometers and piezometers.

\FloatBarrier
\subsubsection{Research and development program}
The R\&D program integrates theoretical and experimental activities: characterization of possible underground sites, and development of sensors and networks to be optimized for the specific site studies. 

The main experimental activities will be
\begin{itemize}
\item{} Acquisition of seismic data at various underground sites in Europe. The data provide information about attenuation of seismic noise with depth, dependence on geology, $e.g.$ sediments, hardrock and salt, and coherence length of signals among different sensors.
\item{} Development of seismic sensors and sensor networks, and their application in GGN monitoring and subtraction schemes. 
\item{} Data analysis of the acquired data for geophysical applications and modeling of the underground sites.
\end{itemize}

Theoretical studies will be carried out in parallel with the experimental activities:
\begin{itemize} 
\item{} Validation of seismic background noise models for specific sites. This work will be done in collaboration with geophysicists. Evaluation of geophysical applications of Einstein Telescope. 
\item{} Analytical modeling of seismic and atmospheric systems, and extraction of estimates for GGN in an underground environment. A set of simplified models, which can be analyzed analytically, will be used.
\item{} Validation of analytical and numerical (finite element) models by using data acquired at underground sites. Analytical estimates will be compared with predictions from finite element models. The finite element approach is the only possible approach for the detailed modeling of a realistic environment.
\item{} Studies of the GGN subtraction problem. The idea here is to subtract the Newtonian noise contributions from the data by using auxiliary measurements. The activity will consist in obtaining estimates about subtraction efficiency, and in the optimization of auxiliary sensors placement configuration. Several scenarios must be taken into account. Firstly, subtraction of stationary noise generated by far away sources. Secondly, GGN subtraction of nearby sources typically associated with non stationary contributions. Wiener and Kalman filtering schemes will be evaluated. 
\end{itemize}

\FloatBarrier
\subsubsection{Coordination of activities}
It is imperative to coordinate the activities of the Einstein Telescope science team members addressed to the identification of the optimal site for the observatory and the consolidation of the infrastructure design. The following activities are planned.

Coordination of site studies and assessment 
\begin{itemize}
\item{} A site assessment process will be instituted. Required site studies will be defined. Particular emphasis will be given to the seismic noise and gravity gradient noise measurements and the local geology. Scientists of candidate countries will be involved in long term site characterization measurement activities. Results will be compared by means of dedicated workshops. 
\item{} A database will be set up and administered to consolidate geo-physical and technical information on possible sites in Europe. 
\item{} Based on the information collected in the site database a refinement process of the list of site candidates to host the Einstein Telescope observatory will be organized. Technical validation will be separated from political selection criteria. 
\item{} A scheme will be implemented to extract site ranking information including cost evaluation. The development of tools that enable trade off processes between infrastructure cost and site parameters versus achievable science will be coordinated. 
The primary goal of the above listed activities is the realization of comprehensive investigations of the geo-physical properties of potential candidate sites as well as the evaluation of the scientific and financial interests of the countries hosting the candidate sites, aiming for a down-selection of the site candidates.
\end{itemize} 

\FloatBarrier
\subsubsection{Consolidation of infrastructure requirements}
It is important to coordinate the consolidation of the vacuum, cryogenics, safety and infrastructure requirements for a third generation gravitational observatory. Most infrastructure aspects of Einstein Telescope will differ significantly from currently operating gravitational wave detectors. We plan to maintain and extend the currently available expertise. This will be achieved by making use not only of the activities within the GEO and Virgo collaborations, but also by networking with various relevant projects within Europe and worldwide. It is of importance that Einstein Telescope continues to monitor and strengthen its connections to the following activities: 
\begin{itemize}
\item{} The realization of second-generation detectors such as Advanced Virgo, Advanced LIGO and GEO-HF and the experience gained from their commissioning will feed directly into the consolidation of the Einstein Telescope infrastructure requirements. 
\item{} Cryogenics are a crucial component for any third generation detector. R\&D from small-scale prototypes in Europe and especially the knowledge gained from the commissioning of the Japanese CLIO and LCGT projects are of high value for Einstein Telescope and will be closely monitored. 
\item{} The infrastructure will be the largest cost factor for a third generation observatory and therefore special emphasis is required for the infra-structure costing. Several large-scale underground projects outside the gravitational wave community such as the Large Hadron Collider at CERN, International Linear Collider, and the Sanford Deep Underground Science and Engineering Laboratory collaborations will be consulted and used as reference to update and improve the cost planning. 
\end{itemize}

Consolidation of vacuum, cryogenics, safety and infrastructure requirements 
\begin{itemize}
\item{} In regular intervals the general requirements for vacuum, cryogenics, safety and infrastructure of the Einstein Telescope observatory will be assessed and updated, according to any new knowledge evolving. 
\item{} The activity of consolidation of the Einstein Telescope design, optimized for each site, will be coordinated. Stimulus will be provided for setting up committees with local scientists and engineers in order to adapt the infrastructure reference design to each individual site and its geological properties. 
\end{itemize}

Modeling of seismic and gravity-gradient noise 
\begin{itemize} 
\item{} Evaluation of gravity gradient noise models and subtraction techniques (based on both analytical and finite-element) will be stimulated. 
\item{} Results will be collected and compared by means of dedicated workshops. 
\end{itemize}

In summary, the sensitivity requirements at low frequencies for third generation interferometers imply the need to extend the candidate site studies, to develop suitable tools (sensors and techniques) for the measurements, analysis and compensation of seismic and Newtonian noise, aimed to define the physical and geological characteristics of candidate underground sites in Europe to host the Einstein Telescope.

%\begin{thebibliography}{1}
%\bibitem{cellaggn} M. Beker $et$ $al$. `{\sl Improving the sensitivity of future GW observatories
%in the 1 - 10 Hz band: Newtonian and seismic noise}', accepted for publication by GRG, 2010.
%\bibitem{et} See the website of Einstein Telescope: http://www.et-gw.eu
%\end{thebibliography}



